\section{Stages of Git}

\begin{frame}
    \frametitle{setting up your environment}
    Let git know who you are\ldots
    \begin{description}
        \item[set user] \hfill \\
            git config -\,-global user.name "Max Mustermann"
        \item[set email] \hfill \\
            git config -\,-global user.email "max.mustermann@example.org"
    \end{description}
\end{frame}

\begin{frame}
    \frametitle{git config}
    global vs. local config
    \begin{itemize}
        \item global config in \$HOME/.gitconfig
        \item repository local config .git/config
    \end{itemize}
    \begin{description}
        \item[aliases] \hfill \\
            git config alias.st=git status
        \item[highlighting] \hfill \\
            git config -\,-global color.ui=always
    \end{description}
\end{frame}

\begin{frame}
    \frametitle{lets start\ldots}
    mkdir my\_new\_project; cd my\_new\_project \\
    git init \\
    or\ldots getting local copy of repository that already exist.\\
    git clone \url{https://github.com/blastmaster/ta-git\_intro.git}
\end{frame}

\begin{frame}
    \frametitle{What is a git repository?}
    \begin{figure}[b]{\textwidth}
        \centering
        \begin{tikzpicture}
            \tikzstyle{state} = [ draw,
                node distance = 1.4em,
                drop shadow = {opacity=0.15},
                font = \fontfamily{lmtt}\selectfont\small,
                shape = rectangle,
                rounded corners = .5em,
                minimum width = 7em,
                minimum height = 2em,
                text opacity = 0.75,
                semithick ]
            \tikzstyle{whatis} = [
                node distance = 1.4em,
                left,
                font = \fontfamily{lmtt}\selectfont\tiny]
            \node[state] (remote) {remote};
            \node[state] (repository) [below=of remote] {.git directory \\
                                                         (Repository)};
            \node[state] (index) [below=of repository] {Staging Area};
            \node[state] (stash) [below=of index] {Stash};
            \node[state] (tree)  [below=of stash] {Working Directory};

            \node[whatis, left] at (remote.west) {Repository on the internet or network.};
            \node[whatis, left] at (repository.west) {Local repository that contains complete history.};
            \node[whatis, left] at (index.west) {Snapshot of the working tree for next commit.};
            \node[whatis, left] at (stash.west) {A place to hide modification if you need a clean workspace.};
            \node[whatis, left] at (tree.west) {The direcotries and files on your filesystem.};
        \end{tikzpicture}
    \end{figure}
\end{frame}

\begin{frame}
    \frametitle{Staging files}
    \begin{figure}[b]{\textwidth}
        \centering
        \begin{tikzpicture}
            \tikzstyle{state} = [ draw,
                node distance = 1.4em,
                drop shadow = {opacity=0.15},
                font = \fontfamily{lmtt}\selectfont\small,
                shape = rectangle,
                rounded corners = .5em,
                minimum width = 7em,
                minimum height = 2em,
                text opacity = 0.75,
                semithick ]
            \tikzstyle{whatis} = [
                node distance = 1.4em,
                font = \fontfamily{lmtt}\selectfont\tiny]
            \node[state] (index) {Staging Area};
            \node[state] (tree)  [below=of index] {Working Directory}
                edge [->, out=0, in=0] node[whatis, right] {git add <filename>} (index)
                edge [<-, out=180, in=180] node[whatis, left] {git reset HEAD <filename>} (index);
        \end{tikzpicture}
    \end{figure}
\end{frame}


% TODO
% add git stash --keep-index
\begin{frame}
    \frametitle{Stashing}
    \begin{figure}[b]{\textwidth}
        \centering
        \begin{tikzpicture}
            \tikzstyle{state} = [ draw,
                node distance = 1.4em,
                drop shadow = {opacity=0.15},
                font = \fontfamily{lmtt}\selectfont\small,
                shape = rectangle,
                rounded corners = .5em,
                minimum width = 7em,
                minimum height = 2em,
                text opacity = 0.75,
                semithick ]
            \tikzstyle{whatis} = [
                node distance = 1.4em,
                font = \fontfamily{lmtt}\selectfont\tiny]
            \node[state] (index) {Staging Area};
            \node[state] (stash) [below=of index] {Stash};
            \node[state] (tree)  [below=of stash] {Working Directory}
                edge [->, out=0, in=0] node [whatis, right] {git stash} (stash)
                edge [<-, out=180, in=180] node [whatis, left] {git apply} (stash);
        \end{tikzpicture}
    \end{figure}
\end{frame}


\begin{frame}
    \frametitle{commiting changes}
    \begin{figure}[b]{\textwidth}
        \centering
        \begin{tikzpicture}
            \tikzstyle{state} = [ draw,
                node distance = 1.4em,
                drop shadow = {opacity=0.15},
                font = \fontfamily{lmtt}\selectfont\small,
                shape = rectangle,
                rounded corners = .5em,
                minimum width = 7em,
                minimum height = 2em,
                text opacity = 0.75,
                semithick ]
            \tikzstyle{whatis} = [
                node distance = 1.4em,
                font = \fontfamily{lmtt}\selectfont\tiny]
            \node[state] (repository)  {local repository};
            \node[state] (index) [below=of repository] {Staging Area}
                edge [->, out=0, in=0] node [whatis, right] {git commit} (repository);
            \node[state] (tree)  [below=of stash] {Working Directory}
                edge [<-, out=180, in=180] node [whatis, left] {git checkout <commitid>} (repository)
                edge [->, out=0, in=0] node[whatis, right] {git add <filename>} (index);
        \end{tikzpicture}
    \end{figure}
\end{frame}

\begin{frame}
    \frametitle{version control happens local} % TODO
    \begin{figure}[b]{\textwidth}
        \centering
        \includegraphics[width=0.9\textwidth]{../img/three_states.png}
    \end{figure}
    % TODO add reference
    % http://www.git-scm.com/book/en/v2/book/01-introduction/images/areas.png
\end{frame}

\begin{frame}
    \frametitle{Lifecycle of our files}
    \begin{figure}[b]{\textwidth}
        \centering
            \includegraphics[width=\textwidth]{../img/file_lifecycle.png}
    \end{figure}
    % TODO add reference
    % http://www.git-scm.com/book/en/v2/Git-Basics-Recording-Changes-to-the-Repository
\end{frame}

\begin{frame}
    \frametitle{remote}
    \begin{figure}
        \begin{subfigure}{\textwidth}
            \centering
            \begin{tikzpicture}
            \tikzstyle{state} = [ draw,
                node distance = 1.4em,
                drop shadow = {opacity=0.15},
                font = \fontfamily{lmtt}\selectfont\small,
                shape = rectangle,
                rounded corners = .5em,
                minimum width = 7em,
                minimum height = 2em,
                text opacity = 0.75,
                semithick ]
            \tikzstyle{whatis} = [
                node distance = 1.4em,
                font = \fontfamily{lmtt}\selectfont\tiny]
                    \node[state] (remote) {remote: origin};
                    \node[state] (working) [below=of remote] {local repository}
                        edge [<-, in=0, out=0]  node[whatis, right] {git pull origin master} (remote)
                        edge [->, in=180, out=180] node[whatis, left] {git push origin master} (remote);
            \end{tikzpicture}
        \end{subfigure}
        \begin{subfigure}[b]{\textwidth}
            \centering
            \includegraphics[width=\textwidth]{../img/show_remote.png}
        \end{subfigure}
    \end{figure}
\end{frame}

\begin{frame}
    \frametitle{adding a remote}
    \begin{figure}
        \begin{subfigure}{\textwidth}
            \centering
            \begin{tikzpicture}
                \tikzstyle{state} = [ draw,
                    node distance = 1.4em,
                    drop shadow = {opacity=0.15},
                    font = \fontfamily{lmtt}\selectfont\small,
                    shape = rectangle,
                    rounded corners = .5em,
                    minimum width = 7em,
                    minimum height = 2em,
                    text opacity = 0.75,
                    semithick ]
                \tikzstyle{whatis} = [
                    node distance = 2.4em,
                    font = \fontfamily{lmtt}\selectfont\tiny]
                \node[state] (remote) {remote: origin};
                \node[state] (remote2) [right=of remote] {remote: github};
                \node[state] (working) [below=of remote, xshift=1.25cm] {local copy}
                    edge [<->, in=180, out=180] node[whatis, left, align=center, midway] {
                    git pull origin master \\
                    git push origin master} (remote)
                    edge [<->, in=0, out=0] node[whatis, right, align=center, midway] {git pull foo master \\
                    git push foo master} (remote2);
            \end{tikzpicture}
        \end{subfigure}
        \begin{subfigure}[b]{\textwidth}
            \centering
            \includegraphics[width=\textwidth]{../img/adding_remote.png}
        \end{subfigure}
    \end{figure}
\end{frame}

\begin{frame}
    \frametitle{Tags}
    Tags are like bookmarks on commits.
    \begin{figure}
        \begin{subfigure}[b]{\textwidth}
            \centering
            \begin{tikzpicture}
                \gitDAG[grow right sep = 2em]
                {
                    A -- B -- C
                };
                \gitbranch
                {master}
                {above=of C}
                {C}
                \gitHEAD
                {above=of master}
                {master}
            \end{tikzpicture}
            \subcaption{\texttt{git tag -a v0.1 A}}
        \end{subfigure}
    \end{figure}
\end{frame}

\begin{frame}
    \frametitle{Tags}
    \begin{figure}
        \begin{subfigure}[b]{\textwidth}
            \centering
            \begin{tikzpicture}
                \gitDAG[grow right sep = 2em]
                {
                    A -- B -- C
                };
                \gitbranch {master}
                {above=of C}
                {C}
                \gittag
                [v0p1]
                {v0.1}
                {above=of A}
                {A}
                \gitHEAD
                {above=of master}
                {master}
            \end{tikzpicture}
            \subcaption{our brand new tag!}
        \end{subfigure}
    \end{figure}
\end{frame}

\begin{frame}
    \frametitle{Branching}
    \begin{figure}[b]{\textwidth}
        \centering
        \begin{tikzpicture}
            \gitDAG[grow right sep = 2em]
            {
                A -- B -- C
            };
            \gitbranch {master}
            {above=of C}
            {C}
            \gittag
            [v0p1]
            {v0.1}
            {above=of A}
            {A}
            \gitHEAD
            {above=of master}
            {master}
        \end{tikzpicture}
        \caption{lets create a new branch}
    \end{figure}
\end{frame}

\begin{frame}
    \frametitle{TIMTOWTDI}
    \begin{itemize}
        \item git branch <branchname>
        \item git checkout -b <branchname>
    \end{itemize}
\end{frame}

\begin{frame}
    \frametitle{Branching}
    \begin{figure}[b]{\textwidth}
        \centering
        \begin{tikzpicture}
            \gitDAG[grow right sep = 2em]
            {
                A -- B -- {
                    C,
                    D -- E,
                }

            };
            \gitbranch {master}
            {above=of C}
            {C}
            \gittag
            [v0p1]
            {v0.1}
            {above=of A}
            {A}
            \gitbranch
            {feature X}
            {above=of E}
            {E}
            \gitHEAD
            {above=of master}
            {master}
        \end{tikzpicture}
        \caption{lets create a new branch}
    \end{figure}
\end{frame}

\begin{frame}
    \frametitle{Rebasing}
    \begin{figure}
        \centering
        \begin{tikzpicture}
            \gitDAG[grow right sep = 2em]
            {
                A -- B -- {
                    C,
                    D -- E,
                }
            };
            \gittag
            [v0p1]
            {v0.1}
            {above=of A}
            {A}
            \gitremotebranch
            [origmaster]
            {origin/master}
            {above=of C}
            {C}
            \gitbranch
            {master}
            {above=of E}
            {E}
            \gitHEAD
            {above=of master}
            {master}
        \end{tikzpicture}
    \end{figure}
\end{frame}

\begin{frame}
    \frametitle{Rebasing}
    \begin{figure}
        \centering
        \begin{tikzpicture}
            \gitDAG[grow right sep = 2em] {
                A -- B -- {
                    C -- D' -- E',
                    {[nodes=unreachable] D -- E},
                }
            };
            \gittag
            [v0p1]
            {v0.1}
            {above=of A}
            {A}
            \gitremotebranch
            [origmaster]
            {origin/master}
            {above=of C}
            {C}
            \gitbranch
            {master}
            {above=of E'}
            {E'}
            \gitHEAD
            {above=of master}
            {master}
            \gitblob
            {above=of B}
            {master}
        \end{tikzpicture}
    \end{figure}
\end{frame}

% demonstrate merging of two branches
\begin{frame}
    \frametitle{merging}
    \begin{figure}
        \begin{subfigure}[b]{\textwidth}
            \centering
            \begin{tikzpicture}
                \gitDAG[grow right sep = 2em]
                {
                    A -- B -- {
                        C,
                        D -- E,
                    }
                };
                \gitbranch
                {master}
                {above=of C}
                {C}
                \gitbranch
                {feature X}
                {above=of E}
                {E}
                \gitHEAD
                {above=of feature X}
                {feature X}
            \end{tikzpicture}
            \subcaption{Before\ldots}
        \end{subfigure}
    \end{figure}
\end{frame}

\begin{frame}
    \frametitle{merging}
    \begin{figure}
        \begin{subfigure}[b]{\textwidth}
            \centering
            \begin{tikzpicture}
                \gitDAG[grow right sep = 2em]
                {
                    A -- B -- {
                        C -- E',
                        {D -- E -- E'},
                    }
                };
                \gitbranch
                {master}
                {above=of E'}
                {E'}
                \gitbranch
                {feature X}
                {below=of E}
                {E}
                \gitHEAD
                {above=of master}
                {master}
            \end{tikzpicture}
            \subcaption{after: \texttt{git merge feature X}}
        \end{subfigure}
    \end{figure}
\end{frame}

\begin{frame}
    \frametitle{follow the changes}
    \begin{itemize}
        \item git diff [-\,-staged]
        \item git log
        \item git blame
    \end{itemize}
\end{frame}

