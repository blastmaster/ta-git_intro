\documentclass[11pt]{beamer}
\usepackage[utf8x]{inputenc}

\usepackage{multirow}
\usepackage{adjustbox}
\usepackage{hyperref}

\usepackage{tikz}

\usetikzlibrary{arrows, backgrounds, positioning, fit, matrix}

\usepackage{gitdags}
\usepackage{color}
\usepackage{listings}

\newif\ifzihbackground
\zihbackgroundtrue
%\zihbackgroundfalse

% Yes, this is dirty
\newcommand\zihmaketitle{
	\definecolor{white}{gray}{1.00}%
	\setbeamercolor{normaltext}{bg=darkblue}%
	\setbeamertemplate{headline}{%
		\vskip6.15mm\color{white}\setlength{\arrayrulewidth}{0.3pt}%
		\begin{tabular*}{\paperwidth}[b]{l@{\extracolsep\fill}}%
			\hspace*{3.0mm}\color{white}%
			\includegraphics[height=7.81mm]{theme/logo/tu_logo_black}\\[1.2mm]%
			\hline\hspace*{11.76mm}\rule[-0.8mm]{0pt}{2.47mm}%
			\def\@@dummyComma{}\rule{0pt}{5.8pt}%
			\insertinstitute \\%
			\hline%
		\end{tabular*}%
		\hspace{-\paperwidth}%
	}%
    \ifzihbackground
      \setbeamertemplate{footline}{}
      \setbeamertemplate{background}{\includegraphics[height=\paperheight,width=\paperwidth]{theme/logo/bg}}
      \else
      \setbeamertemplate{footline}{
          \parbox[t][22mm]{\paperwidth}{
              \vspace*{-8.18mm}
              \rule
              {98.6mm}{0pt}\includegraphics[height=15mm]{theme/logo/zih_logo_white}

          }
      }
    \fi%
   \frame{\titlepage}
    % Kopf-/Fusszeilen fuer restliche Folien
    \setbeamercolor{normal text}{bg=white}
    \setbeamertemplate{background}{}
    \setbeamertemplate{headline}[zih01 theme]
    \setbeamertemplate{footline}[zih01 theme]
}

\usetheme{Dresden}
%\useoutertheme{theme/zih01}
%\useinnertheme{theme/zih01}
\usepackage{theme/beamerouterthemezih01}
\usepackage{theme/beamerinnerthemezih01}

%\useinnertheme{rounded}
\definecolor{darkblue}{rgb}{0.04, 0.16, 0.32}
% font color for headlines etc.
\setbeamercolor*{structure}{fg=darkblue,bg=white}
% disable navigation symbols
\setbeamertemplate{navigation symbols}{}
% can't remember what this is good for
\setbeamercovered{transparent}

% reduce margin size
\setbeamersize{text margin left=0.7cm}
\setbeamersize{text margin right=0.7cm}
%
% Outer Color Theme "whale" sorgt f?r strenge farbliche Trennen zwischen Zierrat
% und dem eigentlichen Inhalt. Ein dunkler Hintergrund f?r den Folientitel wirkt
% aber zu aufdringlich.
%
\usecolortheme{orchid}
%\setbeamercolor{titlelike}{parent=structure}

%
% Inner Color Theme "orchid" sorgt f?r farblich abgesetzt Bl?cke (Definitionen,
% S?tze, Beispiele, Beweise, ...).
%
%\usecolortheme{orchid}

%zum drucken
%\usepackage{pgfpages}
%\pgfpagesuselayout{resize to}[a4paper,border shrink=5mm,port]
%\pgfpagesuselayout{4 on 1}[a4paper,border shrink=3mm, landscape]

%%%%%%%%%%%%%%%%%%%%%%%%%%%%

\definecolor{LightGray}		{gray}{0.9}
\definecolor{Gray}		{gray}{0.5}
\definecolor{DarkGray}     	{gray}{0.2}
\definecolor{listinggray} 	{gray}{0.96}
\definecolor{DarkGreen}     	{rgb}{0.0,0.6,0.0}
\definecolor{DarkRed}     	{rgb}{0.6,0.0,0.0}
\definecolor{DarkBlue}     	{rgb}{0.0,0.0,0.6}
\definecolor{DarkCyan}     	{rgb}{0.7,0.7,0.2}
\definecolor{DarkDarkGreen}	{rgb}{0.0,0.4,0.0}

\lstset{language=C}
\lstset{linewidth=0.99\textwidth}
%\lstset{boxpos=c}
\lstset{xleftmargin=0.03\textwidth}
%\lstset{breaklines=true}
\lstset{framexleftmargin=0.03\textwidth}
\lstset{abovecaptionskip=\smallskipamount}
\lstset{belowcaptionskip=\smallskipamount}
\lstset{basicstyle=\ttfamily\tiny}
\lstset{backgroundcolor=\color{listinggray}}
%\lstset{frameround=ffff}
%\lstset{frame=shadowbox}
%\lstset{rulesepcolor=\color{Gray}}
\lstset{numbers=left}
\lstset{numberstyle=\tiny \color{DarkGray}}
\lstset{numbersep=0.01\textwidth}
\lstset{showstringspaces=false}
%\lstset{showspaces=false}
\lstset{tabsize=4}

%% all words in the following list are printed in bold letters in a listing 
\lstset{emph={__asm__, __volatile__, return, main,},emphstyle={\bfseries\color{DarkGray}}}
\lstset{captionpos=b}

% Style für C Sourcecode
\lstdefinestyle{CA}{
        language=C,
        basicstyle=\ttfamily\scriptsize,
        keywordstyle=\ttfamily\bfseries\color{DarkBlue},
        stringstyle=\ttfamily\color{DarkRed},
        commentstyle=\ttfamily\color{DarkGreen},
        identifierstyle=\ttfamily\color{DarkCyan},
        backgroundcolor=\color{listinggray},
}

%%%%%%%%%%%%%%%%%%%%%%%%%%%%

\newcommand\Small{\fontsize{8}{8.0}\selectfont}
\newcommand*\LSTfont{\Small\ttfamily}


\definecolor{mygreen}{rgb}{0,0.6,0}
\definecolor{mygray}{gray}{0.95}

\lstset{
    language=sh,
    basicstyle=\LSTfont,
    backgroundcolor=\color{mygray},
    showspaces=false,
    showstringspaces=false,
    showtabs=false,
    frame=single,
    rulecolor=\color{black},
    tabsize=4,
    breaklines=true,
    keywordstyle=\color{blue},
    commentstyle=\color{darkgray},
    stringstyle=\color{orange}
}

\date{11. Mai 2017}
\institute[ZIH TUD]{Zentrum f\"{u}r Informationsdienste und Hochleistungsrechnen -- TU Dresden}
\title[LCTP]{Introduction into Git}
\author[Oeste]{Sebastian~Oeste}

\room{FAL 011}
\address{Chemnitzer Strasse 50}
\city{01069 Dresden}
\phone{+49 0351 - 463 32405}
\email{sebastian.oeste@tu-dresden.de}


\date{\today}

\setbeamercovered{transparent}

\begin{document}

\zihmaketitle

%\begin{frame}{Table of Contents}
    %\tableofcontents
%\end{frame}

\section{Version Control Systems}

\begin{frame}[fragile]{What is git?}

    \emph{Git is a decentralized VCS (Version Control System).}

    \begin{minipage}{0.49\textwidth}
        \begin{itemize}
            \item 
            \item bar
            \item baz
        \end{itemize}
    \end{minipage}
    \begin{minipage}{0.49\textwidth}
        \begin{figure}
            \centering
            \includegraphics[height=0.6\textheight]{img/xkcdgit.png}
            \caption{\textit{https://xkcd.com/1597/}}
        \end{figure}
    \end{minipage}

\end{frame}

\begin{frame}[fragile]{Version Control Systems}
    \emph{Local Version Control System}
    \vspace{1cm}

    \begin{minipage}{0.49\textwidth}
        \begin{itemize}
            \item store different versions of a file
            \item simple but error prone
            \item \emph{rcs} - Revsion control system
        \end{itemize}
    \end{minipage}
    \begin{minipage}{0.49\textwidth}
    \begin{figure}
        \centering
        \includegraphics[width=0.9\textwidth]{img/local_vcs.png}
    \end{figure}
    \end{minipage}

\end{frame}

\begin{frame}[fragile]{Version Control System}
    \emph{Centralized Version Control Systems}
    \vspace{1cm}

    \begin{minipage}{0.49\textwidth}
        \begin{itemize}
            \item single server that conatins all the versioned files
            \item clients check out files from that central place
            \item single point of failure
            \item \emph{Subversion}, \emph{CVS}
        \end{itemize}
    \end{minipage}
    \begin{minipage}{0.49\textwidth}
        \begin{figure}
            \centering
            \includegraphics[width=0.9\textwidth]{img/centralized_vcs.png}
        \end{figure}
    \end{minipage}

\end{frame}

\begin{frame}[fragile]{Version Control System}
    \emph{Distributed Version Control System}
    \vspace{1cm}

    \begin{minipage}{0.49\textwidth}
        \begin{itemize}
            \item clients fully mirror the repository
            \item serveral remote repositories possible
            \item \emph{Git}, \emph{Mercurial}, \emph{Barzaar}
        \end{itemize}
    \end{minipage}
    \begin{minipage}{0.49\textwidth}
        \begin{figure}
            \centering
            \includegraphics[width=0.9\textwidth]{img/distributed_vcs.png}
        \end{figure}
    \end{minipage}

\end{frame}

\section{Git Introduction}

\begin{frame}[fragile]{Installing git}

    \textbf{Debian:}
    \begin{lstlisting}
    apt install git
    \end{lstlisting}

    \textbf{Arch Linux:}
    \begin{lstlisting}
    pacman -S git
    \end{lstlisting}

    \textbf{Fedora:}
    \begin{lstlisting}
    dnf install git
    \end{lstlisting}

    \begin{description}
        \item{For Linux:} \url{http://git-scm.com/download/linux}
        \item{For Mac:} \url{http://git-scm.com/download/mac}
        \item{For Windows:} \url{http://git-scm.com/download/win}
    \end{description}

\end{frame}

\begin{frame}[fragile]{How it works}
    \begin{figure}
        \centering
        \includegraphics[width=0.8\textwidth]{img/photo_album.jpg}
        \caption{like a photo album of you work}
    \end{figure}
\end{frame}

\begin{frame}[fragile]{How it works}
    \begin{figure}
        \centering
        \includegraphics[width=0.8\textwidth]{img/snapshotbased.png}
        \caption{storing stream of snapshots over time}
    \end{figure}
\end{frame}

%\begin{frame}[fragile]{How it NOT works}
    %\begin{figure}
        %\centering
        %\includegraphics[width=0.8\textwidth]{img/filebased.png}
        %\caption{storing changes to a base file}
    %\end{figure}
%\end{frame}

\begin{frame}{What is a git repository?}

    \begin{figure}
        \centering
        \includegraphics[width=0.9\textwidth]{img/three_states.png}
    \end{figure}
\end{frame}

%\begin{frame}{Basic Workflow}
    %\begin{enumerate}
        %\item Modify files in your working directory.
        %\item Stage the files, adding snapshots of them to your staging area.
        %\item Commit, stores the snapshot in the staging area permanently in
            %your git directory.
        %\item Push your commits to an remote repository on a server.
    %\end{enumerate}
%\end{frame}

\defverbatim\setusername{%
    \scriptsize
    \verb+git config --global user.name "Max Mustermann"+
    \newline
}
\defverbatim\setusermail{%
    \scriptsize
    \verb+git config --global user.mail "max.mustermann@example.org"+
    \newline
}

%\begin{frame}{Configure Git}

    %\emph{Place where the git configuration can live:}
    %\vspace{1cm}

    %% TODO: take a table for that
    %\begin{itemize}
            %\item \textbf{/etc/gitconfig} system wide configuration
                %\textit{-\,-system}
            %\item \textbf{$\sim$/.gitconfig} or \textbf{$\sim$/.config/gitconfig}  user wide configuration \textit{-\,-global}
            %\item \textbf{.git/config} repository specific configuration
    %\end{itemize}

%\end{frame}

\begin{frame}[fragile]{Configure Git}
    \textbf{Show your configuration:}
    \begin{lstlisting}
        git config --list
    \end{lstlisting}
    \textbf{Show specific configuration value:}
    \begin{lstlisting}
        git config user.name
    \end{lstlisting}
    \textbf{Define an alias:}
    \begin{lstlisting}
        git config alias.st=git status
    \end{lstlisting}
    \textbf{Enable highlighting:}
    \begin{lstlisting}
        git config --global color.ui=always
    \end{lstlisting}
\end{frame}

\begin{frame}[fragile]{Setup Your Environment}
    \emph{Three essentail configuration values you should have set.}
    \vspace{1cm}

    \textbf{Your name:}
    \begin{lstlisting}
        git config --global user.name "Max Mustermann"
    \end{lstlisting}
    \textbf{Your email address:}
    \begin{lstlisting}
        git config --global user.mail "max@example.org"
    \end{lstlisting}
    \textbf{Your editor:}
    \begin{lstlisting}
        git config --global core.editor "vim"
    \end{lstlisting}
\end{frame}

\begin{frame}[fragile]{Lets Start\ldots}
    \textbf{Start from scratch:}
    \begin{lstlisting}
mkdir my_new_project
cd my_new_project
git init
    \end{lstlisting}
    \textbf{Get a local copy of a repository that already exist.}
    \begin{lstlisting}
git clone https://github.com/blastmaster/ta-git_intro.git
git clone -b <branchname> <GIT_URL>
    \end{lstlisting}
\end{frame}

%\begin{frame}[fragile]{Follow the changes}
    %\textbf{What is the status of your local repo?}
    %\begin{lstlisting}
        %git status
    %\end{lstlisting}
    %\textbf{What happens so far?}
    %\begin{lstlisting}
        %git log
    %\end{lstlisting}
    %\textbf{What has changed?}
    %\begin{lstlisting}
        %git diff [--staged]
    %\end{lstlisting}
    %\textbf{Who has changed?}
    %\begin{lstlisting}
        %git blame
    %\end{lstlisting}
%\end{frame}

%\begin{frame}[fragile]{Ignoring Files}
    %\emph{Ignore files that follow a specific pattern with a \textbf{.gitignore} file}

    %\textit{.gitignore} rules:
    %\begin{itemize}
        %\item Black lines or lines starting with \# are ignored.
        %\item Standard glob pattern work.
        %\item You can start patterns with a forward slash (/) to avoid recusivity.
        %\item You can negate a pattern by starting it with an exclamation point (!).
    %\end{itemize}

    %\vspace{1em}
    %Example:\\
    %\url{https://github.com/github/gitignore}

%\end{frame}

\section{Git local}


\begin{frame}[fragile]{Git Terminology}
    \begin{figure}[b]
        \centering
        \begin{tikzpicture}
            \tikzstyle{state} = [ draw,
                node distance = 1.4em,
                drop shadow = {opacity=0.15},
                font = \fontfamily{lmtt}\selectfont\small,
                shape = rectangle,
                %rounded corners = .5em,
                minimum width = 4cm,
                minimum height = 2em,
                text opacity = 0.75,
                semithick ]
            \tikzstyle{whatis} = [
                node distance = 1.4em,
                left,
                font = \fontfamily{lmtt}\selectfont\tiny]
            \node[state] (remote) {Remote};
            \node[state] (repository) [below=of remote] {Local Repository};
            \node[state] (index) [below=of repository] {Staging Area};
            %\node[state] (stash) [below=of index] {Stash};
            \node[state] (tree)  [below=of index] {Working Directory};

            \node[whatis, left] at (remote.west) {Repository on the internet or network.};
            \node[whatis, left] at (repository.west) {Local repository that contains complete history.};
            \node[whatis, left] at (index.west) {Snapshot of the working tree for next commit.};
            %\node[whatis, left] at (stash.west) {A place to hide modification if you need a clean workspace.};
            \node[whatis, left] at (tree.west) {The direcotries and files on your filesystem.};
        \end{tikzpicture}
    \end{figure}
\end{frame}

\begin{frame}[fragile]{Basic Workflow}
    \begin{enumerate}
        \item work in your \emph{working directory}
        \item \emph{add} changes to \emph{staging area}
        \item \emph{commit} into \emph{local repository}
        \item \emph{push} to \emph{remote}
    \end{enumerate}
\end{frame}

\begin{frame}[fragile]{Staging files}
    \begin{figure}[b]
        \centering
        \begin{tikzpicture}
            \tikzstyle{state} = [ draw,
                node distance = 1.4em,
                drop shadow = {opacity=0.15},
                font = \fontfamily{lmtt}\selectfont\small,
                shape = rectangle,
                %rounded corners = .5em,
                minimum height = 2em,
                minimum width = 4cm,
                text opacity = 0.75,
                semithick ]
            \tikzstyle{whatis} = [
                node distance = 1.4em,
                font = \fontfamily{lmtt}\selectfont\tiny]
            \node[state] (index) {Staging Area};
            \node[state] (tree)  [below=of index] {Working Directory}
                edge [->, out=0, in=0] node[whatis, right] {git add <filename>} (index)
                edge [<-, out=180, in=180] node[whatis, left] {git reset HEAD <filename>} (index);
        \end{tikzpicture}
    \end{figure}
\end{frame}

\begin{frame}[fragile]{Undoing things}

    \textbf{Unstating a Staged File:}
    \begin{lstlisting}
        git reset HEAD <file>
    \end{lstlisting}
    \textbf{Unmodifying a Modified File:}
    \begin{lstlisting}
        git checkout -- <file>
    \end{lstlisting}

\end{frame}

%\begin{frame}[fragile]{Using the Stash}
    %\begin{figure}[b]
        %\centering
        %\begin{tikzpicture}
            %\tikzstyle{state} = [ draw,
                %node distance = 1.4em,
                %drop shadow = {opacity=0.15},
                %font = \fontfamily{lmtt}\selectfont\small,
                %shape = rectangle,
                %%rounded corners = .5em,
                %minimum height = 2em,
                %minimum width = 4cm,
                %text opacity = 0.75,
                %semithick ]
            %\tikzstyle{whatis} = [
                %node distance = 1.4em,
                %font = \fontfamily{lmtt}\selectfont\tiny]
            %\node[state] (stash) [below=of index] {Stash};
            %\node[state] (tree)  [below=of stash] {Working Directory}
                %edge [->, out=0, in=0] node [whatis, right] {git stash} (stash)
                %edge [<-, out=180, in=180] node [whatis, left] {git stash apply} (stash);
        %\end{tikzpicture}
    %\end{figure}
%\end{frame}

\begin{frame}[fragile]{Commit Changes}
    \begin{figure}[b]
        \centering
        \begin{tikzpicture}
            \tikzstyle{state} = [ draw,
                node distance = 1.4em,
                drop shadow = {opacity=0.15},
                font = \fontfamily{lmtt}\selectfont\small,
                shape = rectangle,
                %rounded corners = .5em,
                minimum height = 2em,
                minimum width = 4cm,
                text opacity = 0.75,
                semithick ]
            \tikzstyle{whatis} = [
                node distance = 1.4em,
                font = \fontfamily{lmtt}\selectfont\tiny]
            \node[state] (repository)  {Local Repository};
            \node[state] (index) [below=of repository] {Staging Area}
                edge [->, out=0, in=0] node [whatis, right] {git commit} (repository);
            \node[state] (tree)  [below=of index] {Working Directory}
                edge [<-, out=180, in=180] node [whatis, left] {git checkout <commitid>} (repository)
                edge [->, out=0, in=0] node[whatis, right] {git add <filename>} (index);
        \end{tikzpicture}
    \end{figure}
\end{frame}


\section{Remotes}

\begin{frame}[fragile]{Working Remotes}
    \begin{figure}[b]
        \centering
        \begin{tikzpicture}
            \tikzstyle{state} = [ draw,
                node distance = 1.4em,
                drop shadow = {opacity=0.15},
                font = \fontfamily{lmtt}\selectfont\small,
                shape = rectangle,
                %rounded corners = .5em,
                minimum height = 2em,
                minimum width = 4cm,
                text opacity = 0.75,
            semithick ]
            \tikzstyle{whatis} = [
                node distance = 1.4em,
                font = \fontfamily{lmtt}\selectfont\tiny]
                \node[state] (remote) {Remote: origin};
                \node[state] (working) [below=of remote] {Local Repository}
                edge [<-, in=0, out=0]  node[whatis, right] {git fetch origin master} (remote)
                edge [->, in=180, out=180] node[whatis, left] {git push origin master} (remote);
        \end{tikzpicture}
    \end{figure}
\end{frame}


%\defverbatim\myadd{%
    %\scriptsize
    %\verb+git remote add github git@github.com:username/repo.git+
    %\newline
%}

%\begin{frame}[fragile]{Working with Remotes}
    %\myadd
    %\begin{figure}
        %\centering
        %\begin{tikzpicture}
            %\tikzstyle{state} = [ draw,
                %node distance = 1.4em,
                %drop shadow = {opacity=0.15},
                %font = \fontfamily{lmtt}\selectfont\small,
                %shape = rectangle,
                %%rounded corners = .5em,
                %minimum height = 2em,
                %minimum width = 4cm,
                %text opacity = 0.75,
            %semithick ]
            %\tikzstyle{whatis} = [
                %node distance = 2.4em,
                %font = \fontfamily{lmtt}\selectfont\tiny]
                %\node[state] (remote) {remote: origin};
                %\node[state] (remote2) [right=of remote] {remote: github};
                %\node[state] (working) [below=of remote, xshift=1.25cm] {local copy}
                %edge [<->, in=180, out=180] node[whatis, left, align=center] {
                    %git pull origin master \\
                %git push origin master} (remote)
                %edge [<->, in=0, out=0] node[whatis, right, align=center] {git pull github master \\
                %git push github master} (remote2);
        %\end{tikzpicture}
    %\end{figure}
%\end{frame}

\begin{frame}[fragile]{Working with Remotes}
    \textbf{Showing your remotes:}
    \begin{lstlisting}
        git remote -v
    \end{lstlisting}
    \textbf{Showing remote information:}
    \begin{lstlisting}
        git remote show <remote>
    \end{lstlisting}
    \textbf{Rename a remote:}
    \begin{lstlisting}
        git remote rename <oldname> <newname>
    \end{lstlisting}
    \textbf{Remove a remote:}
    \begin{lstlisting}
        git remote rm <remote>
    \end{lstlisting}

\end{frame}

%\begin{frame}{Recording Changes}
    %\emph{Each file in your working directory can be in one of two states:
        %tracked or untracked.}

    %\begin{figure}
        %\centering
            %\includegraphics[width=\textwidth]{img/file_lifecycle.png}
    %\end{figure}

%\end{frame}

\begin{frame}[fragile]{Follow the changes}
    \textbf{What is the status of your local repo?}
    \begin{lstlisting}
        git status
    \end{lstlisting}
    \textbf{What happens so far?}
    \begin{lstlisting}
        git log
    \end{lstlisting}
    \textbf{What has changed?}
    \begin{lstlisting}
        git diff [--staged]
    \end{lstlisting}
    \textbf{Who has changed?}
    \begin{lstlisting}
        git blame <file>
    \end{lstlisting}
\end{frame}

%\begin{frame}[fragile]{Working with the log}
    %\centering
    %\adjustbox{max height=\dimexpr\textheight-5.5cm\relax,
               %max width=\textwidth}{
    %\begin{tabular}{ll}
        %\textbf{Option} & \textbf{Description} \\ \hline
        %\textit{-(n)} & Show only the last $n$ commits\\ \hline
        %\textit{-\,-since, -\,-after} & Limit the commits to those made after the specified date.\\ \hline
        %\textit{-\,-unitl, -\,-before} & Limit the commit to those made before the specified date.\\ \hline
        %\textit{-\,-author} & Only show commits in which the author entry matches specified string.\\ \hline
        %\textit{-\,-committer} & Only show commits in which the committer entry matches the specified string.\\ \hline
        %\textit{-\,-grep} & Only show commits with a commit message containing the string.\\ \hline
        %\textit{-S}  & Only show commits adding or removing code matching the string.\\ \hline
    %\end{tabular}
    %}
%\end{frame}

%%% BRANCHING %%%

\begin{frame}[fragile]{Branching - master branch}
    \begin{minipage}{0.49\textwidth}
        \begin{figure}
            \begin{tikzpicture}
                \visible<-1>{
                    \gitDAG[grow right sep = 2em]
                    {
                        A
                    };
                    \gitbranch {master}
                    {below=of A}
                    {A}
                    \gitHEAD
                    {above=of A}
                    {A}
                }
                \visible<2>{
                    \gitDAG[grow right sep = 2em]
                    {
                        A -- B
                    };
                    \gitbranch {master}
                    {below=of B}
                    {B}
                    \gitHEAD
                    {above=of B}
                    {B}
                }
            \end{tikzpicture}
        \end{figure}
    \end{minipage}
    \begin{minipage}{0.49\textwidth}
        \begin{itemize}
            \visible<1->{
                \item \texttt{git add}
                \item \texttt{git commit}
            }
            \visible<2->{
                \item \texttt{git add}
                \item \texttt{git commit}
            }
        \end{itemize}
    \end{minipage}
\end{frame}

\section{Branches}

\begin{frame}[fragile]{Branching - a new branch}
    \begin{minipage}{0.59\textwidth}
    \begin{figure}[b]
        \centering
        \begin{tikzpicture}
            \visible<-1>
            {
                \gitDAG[grow right sep = 2em]
                {
                    A -- B
                };
                \gitbranch {master}
                {below=of B}
                {B}
                \gitHEAD
                {above=of B}
                {B}
            }
            \visible<2->
            {
                %TODO
                \gitDAG[grow right sep = 2em]
                {
                    A -- B -- {
                        D,
                    }
                };
                \gitbranch {master}
                {below=of B}
                {B}
                \gitbranch
                {feature-X}
                {below=of D}
                {D}
                \gitHEAD
                {above=of D}
                {D}
            }
        \end{tikzpicture}
    \end{figure}
    \end{minipage}
    \begin{minipage}{0.39\textwidth}
        \begin{itemize}
                \visible<1->
                {
                    \item \texttt{git branch feature-X}
                    \item \texttt{git checkout feature-X}
                }
                \visible<2->
                {
                    \item \texttt{git add}
                    \item \texttt{git commit}
                }
        \end{itemize}
    \end{minipage}
\end{frame}

\begin{frame}[fragile]{Branching - a new branch}
    \begin{minipage}{0.59\textwidth}
        \begin{figure}[b]
            \centering
            \begin{tikzpicture}
                \visible<1>
                {
                    \gitDAG[grow right sep = 2em]
                    {
                        A -- B -- {
                            C,
                            D,
                        }

                    };
                    \gitbranch {master}
                    {above=of C}
                    {C}
                    \gitbranch
                    {feature-X}
                    {below=of D}
                    {D}
                    \gitHEAD
                    {above=of master}
                    {master}
                }
                \visible<2->
                {
                    \gitDAG[grow right sep = 2em]
                    {
                        A -- B -- {
                            C,
                            D -- E,
                        }

                    };
                    \gitbranch {master}
                    {above=of C}
                    {C}
                    \gitbranch
                    {feature-X}
                    {below=of E}
                    {E}
                    \gitHEAD
                    {above=of E}
                    {E}
                }
            \end{tikzpicture}
        \end{figure}
    \end{minipage}
    \begin{minipage}{0.39\textwidth}
        \begin{itemize}
                \visible<1>
                {
                    \item \texttt{git checkout master}
                    \item \texttt{git add}
                    \item \texttt{git commit}
                }
                \visible<2->
                {
                    \item \texttt{git checkout feature-X}
                    \item \texttt{git add}
                    \item \texttt{git commit}
                }
        \end{itemize}
    \end{minipage}
\end{frame}

\begin{frame}[fragile]{Branch Syntax}
    \textbf{create new branch:}
        \begin{lstlisting}
            git branch <branchname>
        \end{lstlisting}
        \begin{lstlisting}
            git checkout -b <branchname>
        \end{lstlisting}
    \textbf{delete branch:}
        \begin{lstlisting}
            git branch -d <branchname>
        \end{lstlisting}
    \textbf{list local branches:}
        \begin{lstlisting}
            git branch
            git branch --merged
            git branch --no-merged
        \end{lstlisting}
\end{frame}

% demonstrate merging of two branches
\begin{frame}[fragile]{Merging}
        \begin{figure}
            \centering
            \begin{tikzpicture}
                \gitDAG[grow right sep = 2em]
                {
                    A -- B -- {
                        C,
                        D -- E,
                    }
                };
                \gitbranch
                {master}
                {above=of C}
                {C}
                \gitbranch
                {feature-X}
                {below=of E}
                {E}
                {feature X}
                \gitHEAD
                {above=of master}
                {master}
            \end{tikzpicture}
            \caption{Before merge\ldots}
        \end{figure}
\end{frame}

\begin{frame}[fragile]{Merging}
    \begin{figure}[b]
        \centering
        \begin{tikzpicture}
            \gitDAG[grow right sep = 2em]
            {
                A -- B --{
                C,
                    {D -- E},
                } -- E'
            };
            \gitbranch
            {master}
            {above=of E'}
            {E'}
            \gitbranch
            {feature-X}
            {below=of E}
            {E}
            \gitHEAD
            {above=of master}
            {master}
        \end{tikzpicture}
        \caption{after: \texttt{git merge feature-X}}
    \end{figure}
\end{frame}

\begin{frame}[fragile]{Rebasing}
    \begin{figure}
        \centering
        \begin{tikzpicture}
            \gitDAG[grow right sep = 2em]
            {
                A -- B -- {
                    C,
                    D -- E,
                }
            };
            \gitbranch
            {master}
            {above=of C}
            {C}
            \gitbranch
            {feature-X}
            {below=of E}
            {E}
            \gitHEAD
            {above=of master}
            {master}
        \end{tikzpicture}
        \caption{Before rebase\ldots}
    \end{figure}
\end{frame}

\begin{frame}[fragile]{Rebasing}
    \begin{figure}
        \centering
        \begin{tikzpicture}
            \gitDAG[grow right sep = 2em] {
                A -- B -- {
                    C -- D' -- E',
                    {[nodes=unreachable] D -- E},
                }
            };
            \gitbranch
            {master}
            {above=of E'}
            {E'}
            \gitHEAD
            {above=of master}
            {master}
        \end{tikzpicture}
        \caption{After \texttt{git rebase feature-X}}
    \end{figure}
\end{frame}

%\section{Advanced Usage}

\defverbatim\mytag{%
    \scriptsize
    \verb+git tag -a v0.1 A+
    \newline
}

\begin{frame}[fragile]{Tags}
    Two types of tags
    - lightweight tags (much like a branch that dosen't change, its just a pointer to
    a specific commit.)
    - annotated tags (full stored objects in the git database)
\end{frame}

\begin{frame}
    \frametitle{Tags}
    Tags are like bookmarks on commits.
    \begin{figure}[b]
        \centering
        \begin{tikzpicture}
            \gitDAG[grow right sep = 2em]
            {
                A -- B -- C
            };
            \gitbranch
            {master}
            {above=of C}
            {C}
            \gitHEAD
            {above=of master}
            {master}
        \end{tikzpicture}
        \caption{lets create a tag for commit A}
    \end{figure}
\end{frame}

\begin{frame}[fragile]{Tags}
    \mytag
    \begin{figure}[b]
        \centering
        \begin{tikzpicture}
            \gitDAG[grow right sep = 2em]
            {
                A -- B -- C
            };
            \gitbranch {master}
            {above=of C}
            {C}
            \gittag
            [v0p1]
            {v0.1}
            {above=of A}
            {A}
            \gitHEAD
            {above=of master}
            {master}
        \end{tikzpicture}
    \end{figure}
\end{frame}

\begin{frame}[fragile]{Tags Syntax}
    \textbf{Create tag:}
    \begin{lstlisting}
    git tag -a <commitid>
    \end{lstlisting}
    \textbf{Delete tag:}
    \begin{lstlisting}
    git tag -d <tagname>
    \end{lstlisting}
    \textbf{Filter tag:}
    \begin{lstlisting}
    git tag -l <pattern>
    \end{lstlisting}
    \textbf{Sign tag:}
    \begin{lstlisting}
    git tag -s <tagname>
    \end{lstlisting}
    \textbf{Showing a tag:}
    \begin{lstlisting}
    git show v1.4
    \end{lstlisting}
\end{frame}

\begin{frame}[fragile]{Sharing Tags}
    \textbf{You need to explicitly transfer tags to remote.}
    \begin{lstlisting}
        git push origin <tagname>
    \end{lstlisting}
    or:
    \begin{lstlisting}
        git push origin --tags
    \end{lstlisting}
    \textbf{Checking out Tags:}
    \begin{lstlisting}
        git checkout -b <branchname> <tagname>
    \end{lstlisting}
\end{frame}

\section{Pitfalls}

\begin{frame}[fragile]{Oh shit, I need to change the message on my last commit!}
    \begin{figure}
    \begin{lstlisting}
        git commit --amend
        # follow prompts to change the commit message
    \end{lstlisting}
    \end{figure}
\end{frame}


\begin{frame}[fragile]{Oh shit, I committed and immediately realized I need to make one small change!}
    \begin{figure}
    \begin{lstlisting}
        # make your change
        git add . # or add individual files
        git commit --amend
        # follow prompts to change or keep the commit message
        # now your last commit contains that change!
    \end{lstlisting}
    \end{figure}
\end{frame}


\begin{frame}[fragile]{Oh shit, I accidentally committed something to master that should have been on a brand new branch!}
    \begin{figure}
    \begin{lstlisting}
        # create a new branch from the current state of master
        git branch some-new-branch-name
        # remove the commit from the master branch
        git reset HEAD~ --hard
        git checkout some-new-branch-name
        # your commit lives in this branch now :)
    \end{lstlisting}
    \end{figure}
\end{frame}


\begin{frame}[fragile]{Oh shit, I accidentally committed to the wrong branch!}
    \begin{figure}
    \begin{lstlisting}
        # undo the last commit, but leave the changes available
        git reset HEAD~ --soft
        git stash
        # move to the correct branch
        git checkout name-of-the-correct-branch
        git stash pop
        git add . # or add individual files
        git commit -m "your message here"
        # now your changes are on the correct branch
        \end{lstlisting}
\end{figure}
\end{frame}

\begin{frame}[fragile]{Oh shit, I did something terribly wrong, please tell me git has a magic time machine!?!}
    \begin{figure}
    \begin{lstlisting}
        git reflog
        # you will see a list of every thing you've done in git, across all branches!
        # each one has an index HEAD@{index}
        # find the one before you broke everything
        git reset HEAD@{index}
        # magic time machine
    \end{lstlisting}
    \end{figure}
\end{frame}

\begin{frame}[fragile]{Fuck this noise, I give up.}

    \begin{figure}
        \begin{lstlisting}
        cd ..
        rm -rf git-repo-dir
        git clone https://some.github.url/git-repo-dir.git
        cd git-repo-dir
        \end{lstlisting}
    \end{figure}
\end{frame}


\begin{frame}
    \frametitle{Reference and Resources}
    \begin{itemize}
        \item \url{http://www.git-scm.com/docs} % reference
        \item \url{http://www.git-scm.com/book/en/v2} % book
        \item \url{http://ohshitgit.com/} % oh shit git see pitfalls.tex
        \item \url{https://git.wiki.kernel.org} % kernel wiki
        \item \url{https://sandofsky.com/blog/git-workflow.html} % workflow
        \item \url{http://ndpsoftware.com/git-cheatsheet.html} % cheatcheet
        \item \url{http://yasoob.me/learn-git/} % simple overview
        \item \url{http://learngitbranching.js.org} % online demo to learn branching
        \item \url{https://try.github.io/levels/1/challenges/1} % github online demo
    \end{itemize}
\end{frame}


\begin{frame}{Tools}

    \textbf{Full Clients:}
    \begin{description}
        \item[tig:] \url{https://jonas.github.io/tig/}
        \item[gitk:] \url{https://git-scm.com/docs/gitk}
    \end{description}

    \textbf{Merge / Diff Tools:}
    \begin{description}
        \item[meld:] \url{http://meldmerge.org/}
        \item[kompare:] \url{https://www.kde.org/applications/development/kompare}
    \end{description}

\end{frame}


\begin{frame}{End}
    \begin{minipage}{0.49\textheight}
        \begin{figure}
            \centering
            \includegraphics[width=\textwidth]{img/xkcd_git_commit.png}
            \caption{\textit{https://xkcd.com/1296/}}
        \end{figure}
    \end{minipage}
    \hspace{1cm}
    \begin{minipage}{0.49\textheight}
        \begin{figure}
        \centering
        \includegraphics[width=\textwidth]{img/git-logo.png}
        \end{figure}
    \end{minipage}
\end{frame}


\end{document}
