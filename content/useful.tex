\section{Advanced Usage}

\defverbatim\mytag{%
    \scriptsize
    \verb+git tag -a v0.1 A+
    \newline
}

\begin{frame}
    \frametitle{Tags}
    Tags are like bookmarks on commits.
    \begin{figure}[b]
        \centering
        \begin{tikzpicture}
            \gitDAG[grow right sep = 2em]
            {
                A -- B -- C
            };
            \gitbranch
            {master}
            {above=of C}
            {C}
            \gitHEAD
            {above=of master}
            {master}
        \end{tikzpicture}
    \end{figure}
\end{frame}

\begin{frame}[fragile]{Tags}
    \mytag
    \begin{figure}[b]
        \centering
        \begin{tikzpicture}
            \gitDAG[grow right sep = 2em]
            {
                A -- B -- C
            };
            \gitbranch {master}
            {above=of C}
            {C}
            \gittag
            [v0p1]
            {v0.1}
            {above=of A}
            {A}
            \gitHEAD
            {above=of master}
            {master}
        \end{tikzpicture}
    \end{figure}
\end{frame}

\begin{frame}[fragile]{Tags Syntax}
    \textbf{Create tag:}
    \begin{lstlisting}
    git tag -a <commitid>
    \end{lstlisting}
    \textbf{Delete tag:}
    \begin{lstlisting}
    git tag -d <tagname>
    \end{lstlisting}
    \textbf{Filter tag:}
    \begin{lstlisting}
    git tag -l <pattern>
    \end{lstlisting}
    \textbf{Sign tag:}
    \begin{lstlisting}
    git tag -s <tagname>
    \end{lstlisting}
    \textbf{Showing a tag:}
    \begin{lstlisting}
    git show v1.4
    \end{lstlisting}
\end{frame}

\begin{frame}[fragile]{Sharing Tags}
    \textbf{You need to explicitly transfer tags to remote.}
    \begin{lstlisting}
        git push origin <tagname>
    \end{lstlisting}
    or:
    \begin{lstlisting}
        git push origin --tags
    \end{lstlisting}
    \textbf{Checking out Tags:}
    \begin{lstlisting}
        git checkout -b <branchname> <tagname>
    \end{lstlisting}
\end{frame}


% Submodule

\begin{frame}[fragile]{Submodules}
    \emph{Submodules allow you to include or embed one or more repositories as a
        sub-folder inside another repository.}
    \vspace{1cm}


    \textbf{Adding a new submodule:}
    \begin{lstlisting}
    git submodule add <GIT_URL> <PATH>
    \end{lstlisting}
    \textbf{Update a submodule:}
    \begin{lstlisting}
    git submodule update --remote
    \end{lstlisting}

    The \textbf{.gitmodules} file stores the mapping between the projects url
    and the local subdirectory you've pulled it into.

\end{frame}

\begin{frame}[fragile]{Submodules}
    \emph{Cloning a Repository with Submodules.}
    \vspace{1cm}

    \begin{lstlisting}
    git clone <GIT_URL>
    git submodule init
    git submodule update --recursive
    \end{lstlisting}
    or shorter:
    \begin{lstlisting}
    git clone <GIT_URL>
    git submodule update --init --recursive
    \end{lstlisting}
    or even shorter:
    \begin{lstlisting}
    git clone --recursive <GIT_URL>
    \end{lstlisting}

\end{frame}

\begin{frame}[fragile]{Submodules}
    \emph{To remove a submodule you need to:}
    \vspace{1cm}

    \begin{enumerate}
        \item Delete the relevant line from the \textbf{.gitmodules} file.
        \item Delete the relevant section from \textbf{.git/config}.
        \item Run \textit{git rm -\,-cached path\_to\_submodule} (no trailing slash).
        \item Commit the superproject.
        \item Delete the now untracked submodule files.
    \end{enumerate}
\end{frame}
