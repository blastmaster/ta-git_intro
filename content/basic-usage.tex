\section{Git local}

\begin{frame}{Recording Changes}
    \emph{Each file in your working directory can be in one of two states:
        tracked or untracked.}

    \begin{figure}
        \centering
            \includegraphics[width=\textwidth]{img/file_lifecycle.png}
    \end{figure}

\end{frame}

\begin{frame}[fragile]{Git Terminology}
    \begin{figure}[b]
        \centering
        \begin{tikzpicture}
            \tikzstyle{state} = [ draw,
                node distance = 1.4em,
                drop shadow = {opacity=0.15},
                font = \fontfamily{lmtt}\selectfont\small,
                shape = rectangle,
                %rounded corners = .5em,
                minimum width = 4cm,
                minimum height = 2em,
                text opacity = 0.75,
                semithick ]
            \tikzstyle{whatis} = [
                node distance = 1.4em,
                left,
                font = \fontfamily{lmtt}\selectfont\tiny]
            \node[state] (remote) {remote};
            \node[state] (repository) [below=of remote] {Local Repository};
            \node[state] (index) [below=of repository] {Staging Area};
            %\node[state] (stash) [below=of index] {Stash};
            \node[state] (tree)  [below=of index] {Working Directory};

            \node[whatis, left] at (remote.west) {Repository on the internet or network.};
            \node[whatis, left] at (repository.west) {Local repository that contains complete history.};
            \node[whatis, left] at (index.west) {Snapshot of the working tree for next commit.};
            %\node[whatis, left] at (stash.west) {A place to hide modification if you need a clean workspace.};
            \node[whatis, left] at (tree.west) {The direcotries and files on your filesystem.};
        \end{tikzpicture}
    \end{figure}
\end{frame}

\begin{frame}[fragile]{Staging files}
    \begin{figure}[b]
        \centering
        \begin{tikzpicture}
            \tikzstyle{state} = [ draw,
                node distance = 1.4em,
                drop shadow = {opacity=0.15},
                font = \fontfamily{lmtt}\selectfont\small,
                shape = rectangle,
                %rounded corners = .5em,
                minimum height = 2em,
                minimum width = 4cm,
                text opacity = 0.75,
                semithick ]
            \tikzstyle{whatis} = [
                node distance = 1.4em,
                font = \fontfamily{lmtt}\selectfont\tiny]
            \node[state] (index) {Staging Area};
            \node[state] (tree)  [below=of index] {Working Directory}
                edge [->, out=0, in=0] node[whatis, right] {git add <filename>} (index)
                edge [<-, out=180, in=180] node[whatis, left] {git reset HEAD <filename>} (index);
        \end{tikzpicture}
    \end{figure}
\end{frame}

%\begin{frame}[fragile]{Using the Stash}
    %\begin{figure}[b]
        %\centering
        %\begin{tikzpicture}
            %\tikzstyle{state} = [ draw,
                %node distance = 1.4em,
                %drop shadow = {opacity=0.15},
                %font = \fontfamily{lmtt}\selectfont\small,
                %shape = rectangle,
                %%rounded corners = .5em,
                %minimum height = 2em,
                %minimum width = 4cm,
                %text opacity = 0.75,
                %semithick ]
            %\tikzstyle{whatis} = [
                %node distance = 1.4em,
                %font = \fontfamily{lmtt}\selectfont\tiny]
            %\node[state] (stash) [below=of index] {Stash};
            %\node[state] (tree)  [below=of stash] {Working Directory}
                %edge [->, out=0, in=0] node [whatis, right] {git stash} (stash)
                %edge [<-, out=180, in=180] node [whatis, left] {git stash apply} (stash);
        %\end{tikzpicture}
    %\end{figure}
%\end{frame}

\begin{frame}[fragile]{Commit Changes}
    \begin{figure}[b]
        \centering
        \begin{tikzpicture}
            \tikzstyle{state} = [ draw,
                node distance = 1.4em,
                drop shadow = {opacity=0.15},
                font = \fontfamily{lmtt}\selectfont\small,
                shape = rectangle,
                %rounded corners = .5em,
                minimum height = 2em,
                minimum width = 4cm,
                text opacity = 0.75,
                semithick ]
            \tikzstyle{whatis} = [
                node distance = 1.4em,
                font = \fontfamily{lmtt}\selectfont\tiny]
            \node[state] (repository)  {local repository};
            \node[state] (index) [below=of repository] {Staging Area}
                edge [->, out=0, in=0] node [whatis, right] {git commit} (repository);
            \node[state] (tree)  [below=of index] {Working Directory}
                edge [<-, out=180, in=180] node [whatis, left] {git checkout <commitid>} (repository)
                edge [->, out=0, in=0] node[whatis, right] {git add <filename>} (index);
        \end{tikzpicture}
    \end{figure}
\end{frame}

\begin{frame}[fragile]{Undoing things}

    \textbf{Unstating a Staged File:}
    \begin{lstlisting}
        git reset HEAD <file>
    \end{lstlisting}
    \textbf{Unmodifying a Modified File:}
    \begin{lstlisting}
        git checkout -- <file>
    \end{lstlisting}

\end{frame}

\section{Remotes}

\begin{frame}[fragile]{Working Remotes}
    \begin{figure}[b]
        \centering
        \begin{tikzpicture}
            \tikzstyle{state} = [ draw,
                node distance = 1.4em,
                drop shadow = {opacity=0.15},
                font = \fontfamily{lmtt}\selectfont\small,
                shape = rectangle,
                %rounded corners = .5em,
                minimum height = 2em,
                minimum width = 4cm,
                text opacity = 0.75,
            semithick ]
            \tikzstyle{whatis} = [
                node distance = 1.4em,
                font = \fontfamily{lmtt}\selectfont\tiny]
                \node[state] (remote) {remote: origin};
                \node[state] (working) [below=of remote] {local repository}
                edge [<-, in=0, out=0]  node[whatis, right] {git pull origin master} (remote)
                edge [->, in=180, out=180] node[whatis, left] {git push origin master} (remote);
        \end{tikzpicture}
    \end{figure}
\end{frame}


%\defverbatim\myadd{%
    %\scriptsize
    %\verb+git remote add github git@github.com:username/repo.git+
    %\newline
%}

%\begin{frame}[fragile]{Working with Remotes}
    %\myadd
    %\begin{figure}
        %\centering
        %\begin{tikzpicture}
            %\tikzstyle{state} = [ draw,
                %node distance = 1.4em,
                %drop shadow = {opacity=0.15},
                %font = \fontfamily{lmtt}\selectfont\small,
                %shape = rectangle,
                %%rounded corners = .5em,
                %minimum height = 2em,
                %minimum width = 4cm,
                %text opacity = 0.75,
            %semithick ]
            %\tikzstyle{whatis} = [
                %node distance = 2.4em,
                %font = \fontfamily{lmtt}\selectfont\tiny]
                %\node[state] (remote) {remote: origin};
                %\node[state] (remote2) [right=of remote] {remote: github};
                %\node[state] (working) [below=of remote, xshift=1.25cm] {local copy}
                %edge [<->, in=180, out=180] node[whatis, left, align=center] {
                    %git pull origin master \\
                %git push origin master} (remote)
                %edge [<->, in=0, out=0] node[whatis, right, align=center] {git pull github master \\
                %git push github master} (remote2);
        %\end{tikzpicture}
    %\end{figure}
%\end{frame}

\begin{frame}[fragile]{Working with Remotes}
    \textbf{Showing your remotes:}
    \begin{lstlisting}
        git remote -v
    \end{lstlisting}
    \textbf{Showing remote information:}
    \begin{lstlisting}
        git remote show <remote>
    \end{lstlisting}
    \textbf{Rename a remote:}
    \begin{lstlisting}
        git remote rename <oldname> <newname>
    \end{lstlisting}
    \textbf{Remove a remote:}
    \begin{lstlisting}
        git remote rm <remote>
    \end{lstlisting}

\end{frame}


\begin{frame}[fragile]{Commit history}
    \begin{minipage}{0.49\textwidth}
        \begin{figure}
            \begin{tikzpicture}
                \visible<-1>{
                    \gitDAG[grow right sep = 2em]
                    {
                        A
                    };
                    \gitbranch {master}
                    {below=of A}
                    {A}
                    \gitHEAD
                    {above=of A}
                    {A}
                }
                \visible<2>{
                    \gitDAG[grow right sep = 2em]
                    {
                        A -- B
                    };
                    \gitbranch {master}
                    {below=of A}
                    {A}
                    \gitHEAD
                    {above=of B}
                    {B}
                }
            \end{tikzpicture}
        \end{figure}
    \end{minipage}
    \begin{minipage}{0.49\textwidth}
        \begin{itemize}
            \visible<1->{
                \item \texttt{git add}
                \item \texttt{git commit}
            }
            \visible<2->{
                \item \texttt{git add}
                \item \texttt{git commit}
            }
        \end{itemize}
    \end{minipage}
\end{frame}

\section{Branches}

\begin{frame}[fragile]{Branching}
    \begin{minipage}{0.59\textwidth}
    \begin{figure}[b]
        \centering
        \begin{tikzpicture}
            \visible<-1>
            {
                \gitDAG[grow right sep = 2em]
                {
                    A -- B
                };
                \gitbranch {master}
                {below=of A}
                {A}
                \gitHEAD
                {above=of B}
                {B}
            }
            \visible<2->
            {
                %TODO
                \gitDAG[grow right sep = 2em]
                {
                    A -- B -- {
                        D,
                    }
                };
                \gitbranch {master}
                {below=of A}
                {A}
                \gitbranch
                {feature-X}
                {below=of D}
                {D}
                \gitHEAD
                {above=of D}
                {D}
            }
        \end{tikzpicture}
    \end{figure}
    \end{minipage}
    \begin{minipage}{0.39\textwidth}
        \begin{itemize}
                \item \texttt{git branch feature-X}
                \item \texttt{git checkout feature-X}
                \item \texttt{git add}
                \item \texttt{git commit}
        \end{itemize}
    \end{minipage}
\end{frame}

\begin{frame}[fragile]{Branching}
    \begin{minipage}{0.59\textwidth}
        \begin{figure}[b]
            \centering
            \begin{tikzpicture}
                \visible<1>
                {
                    \gitDAG[grow right sep = 2em]
                    {
                        A -- B -- {
                            C,
                            D,
                        }

                    };
                    \gitbranch {master}
                    {below=of A}
                    {A}
                    \gitbranch
                    {feature-X}
                    {below=of D}
                    {D}
                    \gitHEAD
                    {above=of C}
                    {C}
                }
                \visible<2->
                {
                    \gitDAG[grow right sep = 2em]
                    {
                        A -- B -- {
                            C,
                            D -- E,
                        }

                    };
                    \gitbranch {master}
                    {below=of A}
                    {A}
                    \gitbranch
                    {feature-X}
                    {below=of D}
                    {D}
                    \gitHEAD
                    {above=of E}
                    {E}
                }
            \end{tikzpicture}
        \end{figure}
    \end{minipage}
    \begin{minipage}{0.39\textwidth}
        \begin{itemize}
                \visible<1>
                {
                    \item \texttt{git checkout master}
                    \item \texttt{git add}
                    \item \texttt{git commit}
                }
                \visible<2->
                {
                    \item \texttt{git checkout feature-X}
                    \item \texttt{git add}
                    \item \texttt{git commit}
                }
        \end{itemize}
    \end{minipage}
\end{frame}

\begin{frame}[fragile]{Branch Syntax}
    \textbf{create new branch:}
        \begin{lstlisting}
            git branch <branchname>
        \end{lstlisting}
        \begin{lstlisting}
            git checkout -b <branchname>
        \end{lstlisting}
    \textbf{delete branch:}
        \begin{lstlisting}
            git branch -d <branchname>
        \end{lstlisting}
    \textbf{list local branches:}
        \begin{lstlisting}
            git branch
            git branch --merged
            git branch --no-merged
        \end{lstlisting}
\end{frame}

% demonstrate merging of two branches
\begin{frame}[fragile]{Merging}
        \begin{figure}
            \centering
            \begin{tikzpicture}
                \gitDAG[grow right sep = 2em]
                {
                    A -- B -- {
                        C,
                        D -- E,
                    }
                };
                \gitbranch
                {master}
                {below=of A}
                {A}
                \gitbranch
                {feature-X}
                {below=of D}
                {D}
                {feature X}
                \gitHEAD
                {above=of C}
                {C}
            \end{tikzpicture}
            \caption{Before merge\ldots}
        \end{figure}
\end{frame}

\begin{frame}[fragile]{Merging}
    \begin{figure}[b]
        \centering
        \begin{tikzpicture}
            \gitDAG[grow right sep = 2em]
            {
                A -- B --{
                C,
                    {D -- E},
                } -- E'
            };
            \gitbranch
            {master}
            {below=of A}
            {A}
            \gitbranch
            {feature-X}
            {below=of D}
            {D}
            \gitHEAD
            {above=of E'}
            {E'}
        \end{tikzpicture}
        \caption{after: \texttt{git merge feature-X}}
    \end{figure}
\end{frame}

\begin{frame}[fragile]{Rebasing}
    \begin{figure}
        \centering
        \begin{tikzpicture}
            \gitDAG[grow right sep = 2em]
            {
                A -- B -- {
                    C,
                    D -- E,
                }
            };
            \gitbranch
            {master}
            {below=of A}
            {A}
            \gitbranch
            {feature-X}
            {below=of D}
            {D}
            \gitHEAD
            {above=of C}
            {C}
        \end{tikzpicture}
        \caption{Before rebase\ldots}
    \end{figure}
\end{frame}

\begin{frame}[fragile]{Rebasing}
    \begin{figure}
        \centering
        \begin{tikzpicture}
            \gitDAG[grow right sep = 2em] {
                A -- B -- {
                    C -- D' -- E',
                    {[nodes=unreachable] D -- E},
                }
            };
            \gitbranch
            {master}
            {below=of A}
            {A}
            \gitHEAD
            {above=of E'}
            {E'}
        \end{tikzpicture}
        \caption{After \texttt{git rebase feature-X}}
    \end{figure}
\end{frame}



%\begin{frame}[fragile]{Working with the log}
    %\centering
    %\adjustbox{max height=\dimexpr\textheight-5.5cm\relax,
               %max width=\textwidth}{
    %\begin{tabular}{ll}
        %\textbf{Option} & \textbf{Description} \\ \hline
        %\textit{-(n)} & Show only the last $n$ commits\\ \hline
        %\textit{-\,-since, -\,-after} & Limit the commits to those made after the specified date.\\ \hline
        %\textit{-\,-unitl, -\,-before} & Limit the commit to those made before the specified date.\\ \hline
        %\textit{-\,-author} & Only show commits in which the author entry matches specified string.\\ \hline
        %\textit{-\,-committer} & Only show commits in which the committer entry matches the specified string.\\ \hline
        %\textit{-\,-grep} & Only show commits with a commit message containing the string.\\ \hline
        %\textit{-S}  & Only show commits adding or removing code matching the string.\\ \hline
    %\end{tabular}
    %}
%\end{frame}


\begin{frame}
    \frametitle{Reference and Resources}
    \begin{itemize}
        \item \url{http://www.git-scm.com/docs} % reference
        \item \url{http://www.git-scm.com/book/en/v2} % book
        \item \url{http://ohshitgit.com/} % oh shit git see pitfalls.tex
        \item \url{https://git.wiki.kernel.org} % kernel wiki
        \item \url{https://sandofsky.com/blog/git-workflow.html} % workflow
        \item \url{http://ndpsoftware.com/git-cheatsheet.html} % cheatcheet
        \item \url{http://yasoob.me/learn-git/} % simple overview
        \item \url{http://learngitbranching.js.org} % online demo to learn branching
        \item \url{https://try.github.io/levels/1/challenges/1} % github online demo
    \end{itemize}
\end{frame}


\begin{frame}{Tools}

    \textbf{Full Clients:}
    \begin{description}
        \item[tig:] \url{https://jonas.github.io/tig/}
        \item[gitk:] \url{https://git-scm.com/docs/gitk}
    \end{description}

    \textbf{Merge / Diff Tools:}
    \begin{description}
        \item[meld:] \url{http://meldmerge.org/}
        \item[kompare:] \url{https://www.kde.org/applications/development/kompare}
    \end{description}

\end{frame}
