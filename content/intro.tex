\section{Git Introduction}

\begin{frame}[fragile]{Installing git}

    \textbf{Debian:}
    \begin{lstlisting}
    apt install git
    \end{lstlisting}

    \textbf{Arch Linux:}
    \begin{lstlisting}
    pacman -S git
    \end{lstlisting}

    \textbf{Fedora:}
    \begin{lstlisting}
    dnf install git
    \end{lstlisting}

    \begin{description}
        \item{For Linux:} \url{http://git-scm.com/download/linux}
        \item{For Mac:} \url{http://git-scm.com/download/mac}
        \item{For Windows:} \url{http://git-scm.com/download/win}
    \end{description}

\end{frame}

\begin{frame}[fragile]{How it works}
    \begin{figure}
        \centering
        \includegraphics[width=0.8\textwidth]{img/photo_album.jpg}
        \caption{like a photo album of you work}
    \end{figure}
\end{frame}

\begin{frame}[fragile]{How it works}
    \begin{figure}
        \centering
        \includegraphics[width=0.8\textwidth]{img/snapshotbased.png}
        \caption{storing stream of snapshots over time}
    \end{figure}
\end{frame}

%\begin{frame}[fragile]{How it NOT works}
    %\begin{figure}
        %\centering
        %\includegraphics[width=0.8\textwidth]{img/filebased.png}
        %\caption{storing changes to a base file}
    %\end{figure}
%\end{frame}

\begin{frame}{What is a git repository?}

    \begin{figure}
        \centering
        \includegraphics[width=0.9\textwidth]{img/three_states.png}
    \end{figure}
\end{frame}

%\begin{frame}{Basic Workflow}
    %\begin{enumerate}
        %\item Modify files in your working directory.
        %\item Stage the files, adding snapshots of them to your staging area.
        %\item Commit, stores the snapshot in the staging area permanently in
            %your git directory.
        %\item Push your commits to an remote repository on a server.
    %\end{enumerate}
%\end{frame}

\defverbatim\setusername{%
    \scriptsize
    \verb+git config --global user.name "Max Mustermann"+
    \newline
}
\defverbatim\setusermail{%
    \scriptsize
    \verb+git config --global user.mail "max.mustermann@example.org"+
    \newline
}

%\begin{frame}{Configure Git}

    %\emph{Place where the git configuration can live:}
    %\vspace{1cm}

    %% TODO: take a table for that
    %\begin{itemize}
            %\item \textbf{/etc/gitconfig} system wide configuration
                %\textit{-\,-system}
            %\item \textbf{$\sim$/.gitconfig} or \textbf{$\sim$/.config/gitconfig}  user wide configuration \textit{-\,-global}
            %\item \textbf{.git/config} repository specific configuration
    %\end{itemize}

%\end{frame}

\begin{frame}[fragile]{Configure Git}
    \textbf{Show your configuration:}
    \begin{lstlisting}
        git config --list
    \end{lstlisting}
    \textbf{Show specific configuration value:}
    \begin{lstlisting}
        git config user.name
    \end{lstlisting}
    \textbf{Define an alias:}
    \begin{lstlisting}
        git config alias.st=git status
    \end{lstlisting}
    \textbf{Enable highlighting:}
    \begin{lstlisting}
        git config --global color.ui=always
    \end{lstlisting}
\end{frame}

\begin{frame}[fragile]{Setup Your Environment}
    \emph{Three essentail configuration values you should have set.}
    \vspace{1cm}

    \textbf{Your name:}
    \begin{lstlisting}
        git config --global user.name "Max Mustermann"
    \end{lstlisting}
    \textbf{Your email address:}
    \begin{lstlisting}
        git config --global user.mail "max@example.org"
    \end{lstlisting}
    \textbf{Your editor:}
    \begin{lstlisting}
        git config --global core.editor "vim"
    \end{lstlisting}
\end{frame}

\begin{frame}[fragile]{Lets Start\ldots}
    \textbf{Start from scratch:}
    \begin{lstlisting}
mkdir my_new_project
cd my_new_project
git init
    \end{lstlisting}
    \textbf{Get a local copy of a repository that already exist.}
    \begin{lstlisting}
git clone https://github.com/blastmaster/ta-git_intro.git
git clone -b <branchname> <GIT_URL>
    \end{lstlisting}
\end{frame}

%\begin{frame}[fragile]{Follow the changes}
    %\textbf{What is the status of your local repo?}
    %\begin{lstlisting}
        %git status
    %\end{lstlisting}
    %\textbf{What happens so far?}
    %\begin{lstlisting}
        %git log
    %\end{lstlisting}
    %\textbf{What has changed?}
    %\begin{lstlisting}
        %git diff [--staged]
    %\end{lstlisting}
    %\textbf{Who has changed?}
    %\begin{lstlisting}
        %git blame
    %\end{lstlisting}
%\end{frame}

%\begin{frame}[fragile]{Ignoring Files}
    %\emph{Ignore files that follow a specific pattern with a \textbf{.gitignore} file}

    %\textit{.gitignore} rules:
    %\begin{itemize}
        %\item Black lines or lines starting with \# are ignored.
        %\item Standard glob pattern work.
        %\item You can start patterns with a forward slash (/) to avoid recusivity.
        %\item You can negate a pattern by starting it with an exclamation point (!).
    %\end{itemize}

    %\vspace{1em}
    %Example:\\
    %\url{https://github.com/github/gitignore}

%\end{frame}
